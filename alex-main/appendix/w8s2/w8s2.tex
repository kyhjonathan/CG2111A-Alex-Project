\documentclass[a4paper,12pt,oneside, tikz]{book}  
\usepackage[utf8]{inputenc}
\usepackage{tcolorbox}
\usepackage{amsmath,amssymb,amsthm, enumitem, hyperref, tabto} 
\usepackage[T1]{fontenc}
\usepackage[utf8]{inputenc}
\usepackage[english]{babel}
\usepackage{wrapfig}
\usepackage{lastpage}
\usepackage{tikz}
\usetikzlibrary{external}
\tikzexternalize % activate!
\usepackage[american]{circuitikz}
\usepackage[absolute,overlay]{textpos}
\usepackage[left=2cm,right=2cm]{geometry}
\usepackage[english]{babel}
\usepackage{fancyhdr}
\usepackage{float}
\hypersetup{
    colorlinks=true,
    linkcolor=blue,
    filecolor=magenta,      
    urlcolor=cyan,
    pdftitle={Studio 4},
    pdfpagemode=FullScreen,
    }

 %%%%%%%%%%%%%%%%%%%%%%%%%%%%%%%%%%%%%%%%%%%%%%%%%%%%%%%%%%%%%%%%%%%%%%%%%%%%%%%% 
%%% ~ Arduino Language - Arduino IDE Colors ~                                  %%%
%%%                                                                            %%%
%%% Kyle Rocha-Brownell | 10/2/2017 | No Licence                               %%%
%%% -------------------------------------------------------------------------- %%%
%%%                                                                            %%%
%%% Place this file in your working directory (next to the latex file you're   %%%
%%% working on).  To add it to your project, place:                            %%%
%%%     %%%%%%%%%%%%%%%%%%%%%%%%%%%%%%%%%%%%%%%%%%%%%%%%%%%%%%%%%%%%%%%%%%%%%%%%%%%%%%%% 
%%% ~ Arduino Language - Arduino IDE Colors ~                                  %%%
%%%                                                                            %%%
%%% Kyle Rocha-Brownell | 10/2/2017 | No Licence                               %%%
%%% -------------------------------------------------------------------------- %%%
%%%                                                                            %%%
%%% Place this file in your working directory (next to the latex file you're   %%%
%%% working on).  To add it to your project, place:                            %%%
%%%     %%%%%%%%%%%%%%%%%%%%%%%%%%%%%%%%%%%%%%%%%%%%%%%%%%%%%%%%%%%%%%%%%%%%%%%%%%%%%%%% 
%%% ~ Arduino Language - Arduino IDE Colors ~                                  %%%
%%%                                                                            %%%
%%% Kyle Rocha-Brownell | 10/2/2017 | No Licence                               %%%
%%% -------------------------------------------------------------------------- %%%
%%%                                                                            %%%
%%% Place this file in your working directory (next to the latex file you're   %%%
%%% working on).  To add it to your project, place:                            %%%
%%%    \input{arduinoLanguage.tex}                                             %%%
%%% somewhere before \begin{document} in your latex file.                      %%%
%%%                                                                            %%%
%%% In your document, place your arduino code between:                         %%%
%%%   \begin{lstlisting}[language=Arduino]                                     %%%
%%% and:                                                                       %%%
%%%   \end{lstlisting}                                                         %%%
%%%                                                                            %%%
%%% Or create your own style to add non-built-in functions and variables.      %%%
%%%                                                                            %%%
 %%%%%%%%%%%%%%%%%%%%%%%%%%%%%%%%%%%%%%%%%%%%%%%%%%%%%%%%%%%%%%%%%%%%%%%%%%%%%%%% 

\usepackage{color}
\usepackage{listings}    
\usepackage{courier}

%%% Define Custom IDE Colors %%%
\definecolor{arduinoGreen}    {rgb} {0.17, 0.43, 0.01}
\definecolor{arduinoGrey}     {rgb} {0.47, 0.47, 0.33}
\definecolor{arduinoOrange}   {rgb} {0.8 , 0.4 , 0   }
\definecolor{arduinoBlue}     {rgb} {0.01, 0.61, 0.98}
\definecolor{arduinoDarkBlue} {rgb} {0.0 , 0.2 , 0.5 }

%%% Define Arduino Language %%%
\lstdefinelanguage{Arduino}{
  language=C++, % begin with default C++ settings 
%
%
  %%% Keyword Color Group 1 %%%  (called KEYWORD3 by arduino)
  keywordstyle=\color{arduinoGreen},   
  deletekeywords={  % remove all arduino keywords that might be in c++
                break, case, override, final, continue, default, do, else, for, 
                if, return, goto, switch, throw, try, while, setup, loop, export, 
                not, or, and, xor, include, define, elif, else, error, if, ifdef, 
                ifndef, pragma, warning,
                HIGH, LOW, INPUT, INPUT_PULLUP, OUTPUT, DEC, BIN, HEX, OCT, PI, 
                HALF_PI, TWO_PI, LSBFIRST, MSBFIRST, CHANGE, FALLING, RISING, 
                DEFAULT, EXTERNAL, INTERNAL, INTERNAL1V1, INTERNAL2V56, LED_BUILTIN, 
                LED_BUILTIN_RX, LED_BUILTIN_TX, DIGITAL_MESSAGE, FIRMATA_STRING, 
                ANALOG_MESSAGE, REPORT_DIGITAL, REPORT_ANALOG, SET_PIN_MODE, 
                SYSTEM_RESET, SYSEX_START, auto, int8_t, int16_t, int32_t, int64_t, 
                uint8_t, uint16_t, uint32_t, uint64_t, char16_t, char32_t, operator, 
                enum, delete, bool, boolean, byte, char, const, false, float, double, 
                null, NULL, int, long, new, private, protected, public, short, 
                signed, static, volatile, String, void, true, unsigned, word, array, 
                sizeof, dynamic_cast, typedef, const_cast, struct, static_cast, union, 
                friend, extern, class, reinterpret_cast, register, explicit, inline, 
                _Bool, complex, _Complex, _Imaginary, atomic_bool, atomic_char, 
                atomic_schar, atomic_uchar, atomic_short, atomic_ushort, atomic_int, 
                atomic_uint, atomic_long, atomic_ulong, atomic_llong, atomic_ullong, 
                virtual, PROGMEM,
                Serial, Serial1, Serial2, Serial3, SerialUSB, Keyboard, Mouse,
                abs, acos, asin, atan, atan2, ceil, constrain, cos, degrees, exp, 
                floor, log, map, max, min, radians, random, randomSeed, round, sin, 
                sq, sqrt, tan, pow, bitRead, bitWrite, bitSet, bitClear, bit, 
                highByte, lowByte, analogReference, analogRead, 
                analogReadResolution, analogWrite, analogWriteResolution, 
                attachInterrupt, detachInterrupt, digitalPinToInterrupt, delay, 
                delayMicroseconds, digitalWrite, digitalRead, interrupts, millis, 
                micros, noInterrupts, noTone, pinMode, pulseIn, pulseInLong, shiftIn, 
                shiftOut, tone, yield, Stream, begin, end, peek, read, print, 
                println, available, availableForWrite, flush, setTimeout, find, 
                findUntil, parseInt, parseFloat, readBytes, readBytesUntil, readString, 
                readStringUntil, trim, toUpperCase, toLowerCase, charAt, compareTo, 
                concat, endsWith, startsWith, equals, equalsIgnoreCase, getBytes, 
                indexOf, lastIndexOf, length, replace, setCharAt, substring, 
                toCharArray, toInt, press, release, releaseAll, accept, click, move, 
                isPressed, isAlphaNumeric, isAlpha, isAscii, isWhitespace, isControl, 
                isDigit, isGraph, isLowerCase, isPrintable, isPunct, isSpace, 
                isUpperCase, isHexadecimalDigit, 
                }, 
  morekeywords={   % add arduino structures to group 1
                break, case, override, final, continue, default, do, else, for, 
                if, return, goto, switch, throw, try, while, setup, loop, export, 
                not, or, and, xor, include, define, elif, else, error, if, ifdef, 
                ifndef, pragma, warning,
                }, 
% 
%
  %%% Keyword Color Group 2 %%%  (called LITERAL1 by arduino)
  keywordstyle=[2]\color{arduinoBlue},   
  keywords=[2]{   % add variables and dataTypes as 2nd group  
                HIGH, LOW, INPUT, INPUT_PULLUP, OUTPUT, DEC, BIN, HEX, OCT, PI, 
                HALF_PI, TWO_PI, LSBFIRST, MSBFIRST, CHANGE, FALLING, RISING, 
                DEFAULT, EXTERNAL, INTERNAL, INTERNAL1V1, INTERNAL2V56, LED_BUILTIN, 
                LED_BUILTIN_RX, LED_BUILTIN_TX, DIGITAL_MESSAGE, FIRMATA_STRING, 
                ANALOG_MESSAGE, REPORT_DIGITAL, REPORT_ANALOG, SET_PIN_MODE, 
                SYSTEM_RESET, SYSEX_START, auto, int8_t, int16_t, int32_t, int64_t, 
                uint8_t, uint16_t, uint32_t, uint64_t, char16_t, char32_t, operator, 
                enum, delete, bool, boolean, byte, char, const, false, float, double, 
                null, NULL, int, long, new, private, protected, public, short, 
                signed, static, volatile, String, void, true, unsigned, word, array, 
                sizeof, dynamic_cast, typedef, const_cast, struct, static_cast, union, 
                friend, extern, class, reinterpret_cast, register, explicit, inline, 
                _Bool, complex, _Complex, _Imaginary, atomic_bool, atomic_char, 
                atomic_schar, atomic_uchar, atomic_short, atomic_ushort, atomic_int, 
                atomic_uint, atomic_long, atomic_ulong, atomic_llong, atomic_ullong, 
                virtual, PROGMEM,
                },  
% 
%
  %%% Keyword Color Group 3 %%%  (called KEYWORD1 by arduino)
  keywordstyle=[3]\bfseries\color{arduinoOrange},
  keywords=[3]{  % add built-in functions as a 3rd group
                Serial, Serial1, Serial2, Serial3, SerialUSB, Keyboard, Mouse,
                },      
%
%
  %%% Keyword Color Group 4 %%%  (called KEYWORD2 by arduino)
  keywordstyle=[4]\color{arduinoOrange},
  keywords=[4]{  % add more built-in functions as a 4th group
                abs, acos, asin, atan, atan2, ceil, constrain, cos, degrees, exp, 
                floor, log, map, max, min, radians, random, randomSeed, round, sin, 
                sq, sqrt, tan, pow, bitRead, bitWrite, bitSet, bitClear, bit, 
                highByte, lowByte, analogReference, analogRead, 
                analogReadResolution, analogWrite, analogWriteResolution, 
                attachInterrupt, detachInterrupt, digitalPinToInterrupt, delay, 
                delayMicroseconds, digitalWrite, digitalRead, interrupts, millis, 
                micros, noInterrupts, noTone, pinMode, pulseIn, pulseInLong, shiftIn, 
                shiftOut, tone, yield, Stream, begin, end, peek, read, print, 
                println, available, availableForWrite, flush, setTimeout, find, 
                findUntil, parseInt, parseFloat, readBytes, readBytesUntil, readString, 
                readStringUntil, trim, toUpperCase, toLowerCase, charAt, compareTo, 
                concat, endsWith, startsWith, equals, equalsIgnoreCase, getBytes, 
                indexOf, lastIndexOf, length, replace, setCharAt, substring, 
                toCharArray, toInt, press, release, releaseAll, accept, click, move, 
                isPressed, isAlphaNumeric, isAlpha, isAscii, isWhitespace, isControl, 
                isDigit, isGraph, isLowerCase, isPrintable, isPunct, isSpace, 
                isUpperCase, isHexadecimalDigit, 
                },      
%
%
  %%% Set Other Colors %%%
  stringstyle=\color{arduinoDarkBlue},    
  commentstyle=\color{arduinoGrey},    
%          
%   
  %%%% Line Numbering %%%%
   numbers=left,                    
  numbersep=5pt,                   
  numberstyle=\color{arduinoGrey},    
  %stepnumber=2,                      % show every 2 line numbers
%
%
  %%%% Code Box Style %%%%
  breaklines=true,                    % wordwrapping
  tabsize=2,         
  basicstyle=\ttfamily  
}                                             %%%
%%% somewhere before \begin{document} in your latex file.                      %%%
%%%                                                                            %%%
%%% In your document, place your arduino code between:                         %%%
%%%   \begin{lstlisting}[language=Arduino]                                     %%%
%%% and:                                                                       %%%
%%%   \end{lstlisting}                                                         %%%
%%%                                                                            %%%
%%% Or create your own style to add non-built-in functions and variables.      %%%
%%%                                                                            %%%
 %%%%%%%%%%%%%%%%%%%%%%%%%%%%%%%%%%%%%%%%%%%%%%%%%%%%%%%%%%%%%%%%%%%%%%%%%%%%%%%% 

\usepackage{color}
\usepackage{listings}    
\usepackage{courier}

%%% Define Custom IDE Colors %%%
\definecolor{arduinoGreen}    {rgb} {0.17, 0.43, 0.01}
\definecolor{arduinoGrey}     {rgb} {0.47, 0.47, 0.33}
\definecolor{arduinoOrange}   {rgb} {0.8 , 0.4 , 0   }
\definecolor{arduinoBlue}     {rgb} {0.01, 0.61, 0.98}
\definecolor{arduinoDarkBlue} {rgb} {0.0 , 0.2 , 0.5 }

%%% Define Arduino Language %%%
\lstdefinelanguage{Arduino}{
  language=C++, % begin with default C++ settings 
%
%
  %%% Keyword Color Group 1 %%%  (called KEYWORD3 by arduino)
  keywordstyle=\color{arduinoGreen},   
  deletekeywords={  % remove all arduino keywords that might be in c++
                break, case, override, final, continue, default, do, else, for, 
                if, return, goto, switch, throw, try, while, setup, loop, export, 
                not, or, and, xor, include, define, elif, else, error, if, ifdef, 
                ifndef, pragma, warning,
                HIGH, LOW, INPUT, INPUT_PULLUP, OUTPUT, DEC, BIN, HEX, OCT, PI, 
                HALF_PI, TWO_PI, LSBFIRST, MSBFIRST, CHANGE, FALLING, RISING, 
                DEFAULT, EXTERNAL, INTERNAL, INTERNAL1V1, INTERNAL2V56, LED_BUILTIN, 
                LED_BUILTIN_RX, LED_BUILTIN_TX, DIGITAL_MESSAGE, FIRMATA_STRING, 
                ANALOG_MESSAGE, REPORT_DIGITAL, REPORT_ANALOG, SET_PIN_MODE, 
                SYSTEM_RESET, SYSEX_START, auto, int8_t, int16_t, int32_t, int64_t, 
                uint8_t, uint16_t, uint32_t, uint64_t, char16_t, char32_t, operator, 
                enum, delete, bool, boolean, byte, char, const, false, float, double, 
                null, NULL, int, long, new, private, protected, public, short, 
                signed, static, volatile, String, void, true, unsigned, word, array, 
                sizeof, dynamic_cast, typedef, const_cast, struct, static_cast, union, 
                friend, extern, class, reinterpret_cast, register, explicit, inline, 
                _Bool, complex, _Complex, _Imaginary, atomic_bool, atomic_char, 
                atomic_schar, atomic_uchar, atomic_short, atomic_ushort, atomic_int, 
                atomic_uint, atomic_long, atomic_ulong, atomic_llong, atomic_ullong, 
                virtual, PROGMEM,
                Serial, Serial1, Serial2, Serial3, SerialUSB, Keyboard, Mouse,
                abs, acos, asin, atan, atan2, ceil, constrain, cos, degrees, exp, 
                floor, log, map, max, min, radians, random, randomSeed, round, sin, 
                sq, sqrt, tan, pow, bitRead, bitWrite, bitSet, bitClear, bit, 
                highByte, lowByte, analogReference, analogRead, 
                analogReadResolution, analogWrite, analogWriteResolution, 
                attachInterrupt, detachInterrupt, digitalPinToInterrupt, delay, 
                delayMicroseconds, digitalWrite, digitalRead, interrupts, millis, 
                micros, noInterrupts, noTone, pinMode, pulseIn, pulseInLong, shiftIn, 
                shiftOut, tone, yield, Stream, begin, end, peek, read, print, 
                println, available, availableForWrite, flush, setTimeout, find, 
                findUntil, parseInt, parseFloat, readBytes, readBytesUntil, readString, 
                readStringUntil, trim, toUpperCase, toLowerCase, charAt, compareTo, 
                concat, endsWith, startsWith, equals, equalsIgnoreCase, getBytes, 
                indexOf, lastIndexOf, length, replace, setCharAt, substring, 
                toCharArray, toInt, press, release, releaseAll, accept, click, move, 
                isPressed, isAlphaNumeric, isAlpha, isAscii, isWhitespace, isControl, 
                isDigit, isGraph, isLowerCase, isPrintable, isPunct, isSpace, 
                isUpperCase, isHexadecimalDigit, 
                }, 
  morekeywords={   % add arduino structures to group 1
                break, case, override, final, continue, default, do, else, for, 
                if, return, goto, switch, throw, try, while, setup, loop, export, 
                not, or, and, xor, include, define, elif, else, error, if, ifdef, 
                ifndef, pragma, warning,
                }, 
% 
%
  %%% Keyword Color Group 2 %%%  (called LITERAL1 by arduino)
  keywordstyle=[2]\color{arduinoBlue},   
  keywords=[2]{   % add variables and dataTypes as 2nd group  
                HIGH, LOW, INPUT, INPUT_PULLUP, OUTPUT, DEC, BIN, HEX, OCT, PI, 
                HALF_PI, TWO_PI, LSBFIRST, MSBFIRST, CHANGE, FALLING, RISING, 
                DEFAULT, EXTERNAL, INTERNAL, INTERNAL1V1, INTERNAL2V56, LED_BUILTIN, 
                LED_BUILTIN_RX, LED_BUILTIN_TX, DIGITAL_MESSAGE, FIRMATA_STRING, 
                ANALOG_MESSAGE, REPORT_DIGITAL, REPORT_ANALOG, SET_PIN_MODE, 
                SYSTEM_RESET, SYSEX_START, auto, int8_t, int16_t, int32_t, int64_t, 
                uint8_t, uint16_t, uint32_t, uint64_t, char16_t, char32_t, operator, 
                enum, delete, bool, boolean, byte, char, const, false, float, double, 
                null, NULL, int, long, new, private, protected, public, short, 
                signed, static, volatile, String, void, true, unsigned, word, array, 
                sizeof, dynamic_cast, typedef, const_cast, struct, static_cast, union, 
                friend, extern, class, reinterpret_cast, register, explicit, inline, 
                _Bool, complex, _Complex, _Imaginary, atomic_bool, atomic_char, 
                atomic_schar, atomic_uchar, atomic_short, atomic_ushort, atomic_int, 
                atomic_uint, atomic_long, atomic_ulong, atomic_llong, atomic_ullong, 
                virtual, PROGMEM,
                },  
% 
%
  %%% Keyword Color Group 3 %%%  (called KEYWORD1 by arduino)
  keywordstyle=[3]\bfseries\color{arduinoOrange},
  keywords=[3]{  % add built-in functions as a 3rd group
                Serial, Serial1, Serial2, Serial3, SerialUSB, Keyboard, Mouse,
                },      
%
%
  %%% Keyword Color Group 4 %%%  (called KEYWORD2 by arduino)
  keywordstyle=[4]\color{arduinoOrange},
  keywords=[4]{  % add more built-in functions as a 4th group
                abs, acos, asin, atan, atan2, ceil, constrain, cos, degrees, exp, 
                floor, log, map, max, min, radians, random, randomSeed, round, sin, 
                sq, sqrt, tan, pow, bitRead, bitWrite, bitSet, bitClear, bit, 
                highByte, lowByte, analogReference, analogRead, 
                analogReadResolution, analogWrite, analogWriteResolution, 
                attachInterrupt, detachInterrupt, digitalPinToInterrupt, delay, 
                delayMicroseconds, digitalWrite, digitalRead, interrupts, millis, 
                micros, noInterrupts, noTone, pinMode, pulseIn, pulseInLong, shiftIn, 
                shiftOut, tone, yield, Stream, begin, end, peek, read, print, 
                println, available, availableForWrite, flush, setTimeout, find, 
                findUntil, parseInt, parseFloat, readBytes, readBytesUntil, readString, 
                readStringUntil, trim, toUpperCase, toLowerCase, charAt, compareTo, 
                concat, endsWith, startsWith, equals, equalsIgnoreCase, getBytes, 
                indexOf, lastIndexOf, length, replace, setCharAt, substring, 
                toCharArray, toInt, press, release, releaseAll, accept, click, move, 
                isPressed, isAlphaNumeric, isAlpha, isAscii, isWhitespace, isControl, 
                isDigit, isGraph, isLowerCase, isPrintable, isPunct, isSpace, 
                isUpperCase, isHexadecimalDigit, 
                },      
%
%
  %%% Set Other Colors %%%
  stringstyle=\color{arduinoDarkBlue},    
  commentstyle=\color{arduinoGrey},    
%          
%   
  %%%% Line Numbering %%%%
   numbers=left,                    
  numbersep=5pt,                   
  numberstyle=\color{arduinoGrey},    
  %stepnumber=2,                      % show every 2 line numbers
%
%
  %%%% Code Box Style %%%%
  breaklines=true,                    % wordwrapping
  tabsize=2,         
  basicstyle=\ttfamily  
}                                             %%%
%%% somewhere before \begin{document} in your latex file.                      %%%
%%%                                                                            %%%
%%% In your document, place your arduino code between:                         %%%
%%%   \begin{lstlisting}[language=Arduino]                                     %%%
%%% and:                                                                       %%%
%%%   \end{lstlisting}                                                         %%%
%%%                                                                            %%%
%%% Or create your own style to add non-built-in functions and variables.      %%%
%%%                                                                            %%%
 %%%%%%%%%%%%%%%%%%%%%%%%%%%%%%%%%%%%%%%%%%%%%%%%%%%%%%%%%%%%%%%%%%%%%%%%%%%%%%%% 

\usepackage{color}
\usepackage{listings}    
\usepackage{courier}

%%% Define Custom IDE Colors %%%
\definecolor{arduinoGreen}    {rgb} {0.17, 0.43, 0.01}
\definecolor{arduinoGrey}     {rgb} {0.47, 0.47, 0.33}
\definecolor{arduinoOrange}   {rgb} {0.8 , 0.4 , 0   }
\definecolor{arduinoBlue}     {rgb} {0.01, 0.61, 0.98}
\definecolor{arduinoDarkBlue} {rgb} {0.0 , 0.2 , 0.5 }

%%% Define Arduino Language %%%
\lstdefinelanguage{Arduino}{
  language=C++, % begin with default C++ settings 
%
%
  %%% Keyword Color Group 1 %%%  (called KEYWORD3 by arduino)
  keywordstyle=\color{arduinoGreen},   
  deletekeywords={  % remove all arduino keywords that might be in c++
                break, case, override, final, continue, default, do, else, for, 
                if, return, goto, switch, throw, try, while, setup, loop, export, 
                not, or, and, xor, include, define, elif, else, error, if, ifdef, 
                ifndef, pragma, warning,
                HIGH, LOW, INPUT, INPUT_PULLUP, OUTPUT, DEC, BIN, HEX, OCT, PI, 
                HALF_PI, TWO_PI, LSBFIRST, MSBFIRST, CHANGE, FALLING, RISING, 
                DEFAULT, EXTERNAL, INTERNAL, INTERNAL1V1, INTERNAL2V56, LED_BUILTIN, 
                LED_BUILTIN_RX, LED_BUILTIN_TX, DIGITAL_MESSAGE, FIRMATA_STRING, 
                ANALOG_MESSAGE, REPORT_DIGITAL, REPORT_ANALOG, SET_PIN_MODE, 
                SYSTEM_RESET, SYSEX_START, auto, int8_t, int16_t, int32_t, int64_t, 
                uint8_t, uint16_t, uint32_t, uint64_t, char16_t, char32_t, operator, 
                enum, delete, bool, boolean, byte, char, const, false, float, double, 
                null, NULL, int, long, new, private, protected, public, short, 
                signed, static, volatile, String, void, true, unsigned, word, array, 
                sizeof, dynamic_cast, typedef, const_cast, struct, static_cast, union, 
                friend, extern, class, reinterpret_cast, register, explicit, inline, 
                _Bool, complex, _Complex, _Imaginary, atomic_bool, atomic_char, 
                atomic_schar, atomic_uchar, atomic_short, atomic_ushort, atomic_int, 
                atomic_uint, atomic_long, atomic_ulong, atomic_llong, atomic_ullong, 
                virtual, PROGMEM,
                Serial, Serial1, Serial2, Serial3, SerialUSB, Keyboard, Mouse,
                abs, acos, asin, atan, atan2, ceil, constrain, cos, degrees, exp, 
                floor, log, map, max, min, radians, random, randomSeed, round, sin, 
                sq, sqrt, tan, pow, bitRead, bitWrite, bitSet, bitClear, bit, 
                highByte, lowByte, analogReference, analogRead, 
                analogReadResolution, analogWrite, analogWriteResolution, 
                attachInterrupt, detachInterrupt, digitalPinToInterrupt, delay, 
                delayMicroseconds, digitalWrite, digitalRead, interrupts, millis, 
                micros, noInterrupts, noTone, pinMode, pulseIn, pulseInLong, shiftIn, 
                shiftOut, tone, yield, Stream, begin, end, peek, read, print, 
                println, available, availableForWrite, flush, setTimeout, find, 
                findUntil, parseInt, parseFloat, readBytes, readBytesUntil, readString, 
                readStringUntil, trim, toUpperCase, toLowerCase, charAt, compareTo, 
                concat, endsWith, startsWith, equals, equalsIgnoreCase, getBytes, 
                indexOf, lastIndexOf, length, replace, setCharAt, substring, 
                toCharArray, toInt, press, release, releaseAll, accept, click, move, 
                isPressed, isAlphaNumeric, isAlpha, isAscii, isWhitespace, isControl, 
                isDigit, isGraph, isLowerCase, isPrintable, isPunct, isSpace, 
                isUpperCase, isHexadecimalDigit, 
                }, 
  morekeywords={   % add arduino structures to group 1
                break, case, override, final, continue, default, do, else, for, 
                if, return, goto, switch, throw, try, while, setup, loop, export, 
                not, or, and, xor, include, define, elif, else, error, if, ifdef, 
                ifndef, pragma, warning,
                }, 
% 
%
  %%% Keyword Color Group 2 %%%  (called LITERAL1 by arduino)
  keywordstyle=[2]\color{arduinoBlue},   
  keywords=[2]{   % add variables and dataTypes as 2nd group  
                HIGH, LOW, INPUT, INPUT_PULLUP, OUTPUT, DEC, BIN, HEX, OCT, PI, 
                HALF_PI, TWO_PI, LSBFIRST, MSBFIRST, CHANGE, FALLING, RISING, 
                DEFAULT, EXTERNAL, INTERNAL, INTERNAL1V1, INTERNAL2V56, LED_BUILTIN, 
                LED_BUILTIN_RX, LED_BUILTIN_TX, DIGITAL_MESSAGE, FIRMATA_STRING, 
                ANALOG_MESSAGE, REPORT_DIGITAL, REPORT_ANALOG, SET_PIN_MODE, 
                SYSTEM_RESET, SYSEX_START, auto, int8_t, int16_t, int32_t, int64_t, 
                uint8_t, uint16_t, uint32_t, uint64_t, char16_t, char32_t, operator, 
                enum, delete, bool, boolean, byte, char, const, false, float, double, 
                null, NULL, int, long, new, private, protected, public, short, 
                signed, static, volatile, String, void, true, unsigned, word, array, 
                sizeof, dynamic_cast, typedef, const_cast, struct, static_cast, union, 
                friend, extern, class, reinterpret_cast, register, explicit, inline, 
                _Bool, complex, _Complex, _Imaginary, atomic_bool, atomic_char, 
                atomic_schar, atomic_uchar, atomic_short, atomic_ushort, atomic_int, 
                atomic_uint, atomic_long, atomic_ulong, atomic_llong, atomic_ullong, 
                virtual, PROGMEM,
                },  
% 
%
  %%% Keyword Color Group 3 %%%  (called KEYWORD1 by arduino)
  keywordstyle=[3]\bfseries\color{arduinoOrange},
  keywords=[3]{  % add built-in functions as a 3rd group
                Serial, Serial1, Serial2, Serial3, SerialUSB, Keyboard, Mouse,
                },      
%
%
  %%% Keyword Color Group 4 %%%  (called KEYWORD2 by arduino)
  keywordstyle=[4]\color{arduinoOrange},
  keywords=[4]{  % add more built-in functions as a 4th group
                abs, acos, asin, atan, atan2, ceil, constrain, cos, degrees, exp, 
                floor, log, map, max, min, radians, random, randomSeed, round, sin, 
                sq, sqrt, tan, pow, bitRead, bitWrite, bitSet, bitClear, bit, 
                highByte, lowByte, analogReference, analogRead, 
                analogReadResolution, analogWrite, analogWriteResolution, 
                attachInterrupt, detachInterrupt, digitalPinToInterrupt, delay, 
                delayMicroseconds, digitalWrite, digitalRead, interrupts, millis, 
                micros, noInterrupts, noTone, pinMode, pulseIn, pulseInLong, shiftIn, 
                shiftOut, tone, yield, Stream, begin, end, peek, read, print, 
                println, available, availableForWrite, flush, setTimeout, find, 
                findUntil, parseInt, parseFloat, readBytes, readBytesUntil, readString, 
                readStringUntil, trim, toUpperCase, toLowerCase, charAt, compareTo, 
                concat, endsWith, startsWith, equals, equalsIgnoreCase, getBytes, 
                indexOf, lastIndexOf, length, replace, setCharAt, substring, 
                toCharArray, toInt, press, release, releaseAll, accept, click, move, 
                isPressed, isAlphaNumeric, isAlpha, isAscii, isWhitespace, isControl, 
                isDigit, isGraph, isLowerCase, isPrintable, isPunct, isSpace, 
                isUpperCase, isHexadecimalDigit, 
                },      
%
%
  %%% Set Other Colors %%%
  stringstyle=\color{arduinoDarkBlue},    
  commentstyle=\color{arduinoGrey},    
%          
%   
  %%%% Line Numbering %%%%
   numbers=left,                    
  numbersep=5pt,                   
  numberstyle=\color{arduinoGrey},    
  %stepnumber=2,                      % show every 2 line numbers
%
%
  %%%% Code Box Style %%%%
  breaklines=true,                    % wordwrapping
  tabsize=2,         
  basicstyle=\ttfamily  
}

\urlstyle{same}
\usepackage{xcolor}
\usepackage{colortbl}

\usepackage{listings}
\usepackage{xcolor}
\colorlet{mygray}{black!30}
\colorlet{mygreen}{green!60!blue}
\colorlet{mymauve}{red!60!blue}
\newcommand\inostyle{\lstset{
  backgroundcolor=\color{gray!10},  
  basicstyle=\ttfamily,
  columns=fullflexible,
  breakatwhitespace=false,      
  breaklines=true,                
  captionpos=b,                    
  commentstyle=\color{mygreen}, 
  extendedchars=true,              
  frame=single,                   
  keepspaces=true,
  keywordstyle=\color{blue},      
  language=Arduino,                 
  numbers=none,                
  numbersep=5pt,                   
  numberstyle=\tiny\color{blue}, 
  rulecolor=\color{mygray},        
  showspaces=false,               
  showtabs=false,                 
  stepnumber=5,                  
  stringstyle=\color{mymauve},    
  tabsize=3,                      
  title=\lstname                
}}

% INO environment
\lstnewenvironment{ino}[1][]
{
\inostyle
\lstset{#1}
}
{}


% Ino for external files
\newcommand\inoexternal[2][]{{
\inostyle
\lstinputlisting[#1]{#2}}}

% Ino for inline
\newcommand\inoinline[1]{{\inostyle\lstinline!#1!}}

\usepackage{minted}

\usepackage{graphicx, multicol, latexsym}
\usepackage{blindtext}
\usepackage{subfigure}
\usepackage{caption}
\usepackage{capt-of}
\usepackage{tabu}
\usepackage{booktabs}

\usepackage{fancyhdr}            % Permits header customization. See header section below.
\fancypagestyle{plain}{
    \lhead{}
    \fancyhead[R]{\thepage}
    \fancyhead[L]{}
    \renewcommand{\headrulewidth}{0pt}
    \fancyfoot{}
}

\pagestyle{fancy}
\fancyhead[R]{\thepage}
\fancyhead[L]{}
\renewcommand{\headrulewidth}{0pt}
\fancyfoot{}

\usepackage{array}
\newcolumntype{P}[1]{>{\centering\arraybackslash}p{#1}}

\usepackage{titlesec}

\titleformat{\chapter}[display]{\normalfont\huge\bfseries}{\chaptertitlename\ \thechapter}{20pt}{\Huge}

% this alters "before" spacing (the second length argument) to 0
\titlespacing*{\chapter}{0pt}{0pt}{40pt}


\addto\captionsenglish{\renewcommand{\chaptername}{Activity}} 


\title{\Huge\textbf{Jump Starting ALEX} \\ \Large{CG2111A Studio Report}}

\author{
Kuek Yeau Hao Jonathan (A0258485M), \\
Leong Zhe Ming (A0252060W), \\
Tran Duc Khang (A0242247J), \\ 
Prannaya Gupta (A0242267E)
}

\begin{document}

\maketitle

\newpage

\textbf{Q1. Examine the code in \texttt{w8s2p3.ino} in the Arduino IDE. What do you think you are seeing in the output?} (2 MARKS)

\begin{tcolorbox}
\textbf{Solution:}\\
\texttt{\textbf{58}    char theSize = (char) sizeof(TData); \\
\textbf{59}    Serial.write(theSize); \\
\textbf{60}    Serial.write((char *) &test, sizeof(TData));} \\

\\

The code is trying to print the size of TData, but due to it being converted to a character, this is replaced with 'End of Transmission' (ASCII Character 4). After this, it is printing \texttt{x} and \texttt{y}, but they aren't rendered as ASCII Character 5 is 'Enquiry' and ASCII Character 10 is 'Line Feed', or rather newline.
\end{tcolorbox}

\textbf{Q2. Examine the values of \texttt{x} and \texttt{y} in \texttt{w8s2p3.ino}. Are the values the Pi is receiving correct?} (1 MARK)

\begin{tcolorbox}
The Pi received around \texttt{655365} for \texttt{x} and \texttt{-1092980228} for \texttt{y}, which are clearly incorrect.
\end{tcolorbox}

\textbf{Q3. Is \texttt{x} and \texttt{y} printing correctly now? What do you think caused \texttt{w8s2p3-pi} to print the incorrect answers in Question 1 above?} (3 MARKS)

\begin{tcolorbox}
Yes, it is printing correctly now. \\

The initial code for \texttt{w8s2p3-pi} reads both integers as double the bits, hence \texttt{x} received follows the following formula (following Little Endian):

$$x' = y \times 2^{16} + x = 655365$$

On the other hand, \texttt{y} gives a random value as it is undefined.
\end{tcolorbox}


\textbf{Q4. Take note of the size of \texttt{TData} on the Pi and on the Arduino. Is there a difference in size? Is this difference affecting the answers for \texttt{x} and \texttt{y}?} (3 MARKS)

\begin{tcolorbox}
The size of \texttt{TData} on the Pi is 6 bytes, while the size of \texttt{TData} on the Arduino is 5 bytes. Yes, this is affecting it as \texttt{y} and \texttt{x} are joined when reading from the Pi.
\end{tcolorbox}

\textbf{Q5. Are the \texttt{x}, \texttt{y} and \texttt{c} fields being printed correctly now? What does this tell you about how the \texttt{gcc} compiler on the Pi compile \texttt{TData} versus the compiler on the Arduino?} (3 MARKS)

\begin{tcolorbox}
Yes, it is now printed correctly. \\

This tells us that the \texttt{gcc} compiler pads the structure to keep every structure member of the same size, but the Arduino does no such thing.



\end{tcolorbox}

\begin{ino}

Studio Marks: ___________ / 12
\end{ino}

\end{document}